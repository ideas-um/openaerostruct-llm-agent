\documentclass{article}
\usepackage{graphicx}
\usepackage{amsmath}
\usepackage{geometry}
\geometry{a4paper, margin=1in}

\title{Wing Optimization Report: Analysis of Drag Minimization}
\author{}
\date{January 1, 2026}

\begin{document}
\maketitle

\section*{Introduction}
This report details the analysis of a wing optimization problem aimed at minimizing drag. The initial problem statement by the user was to minimize drag for a wing with a fixed area ($S = 100 \, \text{m}^2$) and span ($b = 10 \, \text{m}$) at a cruise condition corresponding to a lift coefficient ($C_L$) of 2.0. The design variables allowed for complete freedom in the taper, twist, and sweep of the wing. The reformulated problem confirmed these parameters, specifying an objective function to minimize drag, a trim condition of $C_L = 2.0$, geometric constraints of $S = 100 \, \text{m}^2$ and $b = 10 \, \text{m}$, and design variables including taper, twist, and sweep. The optimization was performed using the SLSQP algorithm.

\section*{Optimization Results Analysis}

The optimization process concluded with a 'FAIL' status, indicating that it did not successfully achieve its objectives. The primary constraint, a target $C_L = 2.0$, was not met; the final achieved $C_L$ was only 0.993. Consequently, the objective to minimize drag resulted in a very high final drag coefficient ($C_D = 0.18708912$) at the achieved $C_L$. This $C_D$ value is indicative of a highly inefficient design and a poorly performing optimization.

Several key design variables reached their predefined limits:
\begin{itemize}
    \item \textbf{`wing.taper`}: Reached its lower limit of 0.2, resulting in an extremely narrow wingtip.
    \item \textbf{`wing.twist_cp`}: Reached its upper limit of 10.0 degrees, leading to a constant 10-degree twist across the entire span.
    \item \textbf{`wing.sweep`}: Converged to 6.53 degrees, which is near its lower limit of 5.0 degrees.
    \item \textbf{`alpha` (angle of attack)}: Reached its upper limit of 15.0 degrees. An angle of attack of 15 degrees is excessively high for a typical cruise condition and strongly suggests the wing was struggling to generate the required lift, consistent with the unmet $C_L$ constraint.
\end{itemize}

These extreme design variable values, particularly the aggressive taper, high constant twist, and very high angle of attack, are physically possible but are characteristic of a wing being pushed to its limits to generate lift. Such a configuration inevitably results in very high drag and poor aerodynamic performance. The resultant values are not considered reasonable for an efficient wing design at the specified cruise $C_L$.

The graphical plots, particularly the optimized wing visualization presented in Figure \ref{fig:optimized_wing}, reinforce these observations. The wing geometry clearly shows a highly tapered wing with low sweep, consistent with the design variable values reaching their bounds. The twist plot confirms a constant 10 degrees across the span. Crucially, the lift distribution plot is highly irregular, deviating significantly from the ideal elliptical distribution typically sought for drag minimization. This non-elliptical shape, characterized by multiple peaks and valleys, is a strong indicator of high induced drag and potential flow separation, aligning with the reported high $C_D$ and the optimizer's failure to meet the $C_L$ constraint.

\begin{figure}[h!]
    \centering
    \includegraphics[width=0.8\textwidth]{./Optimized_Wing.pdf}
    \caption{Optimized Wing Geometry, Twist, and Lift Distribution. The figure displays the highly tapered wing, constant twist of 10 degrees, and the highly irregular lift distribution obtained from the optimization.}
    \label{fig:optimized_wing}
\end{figure}

\section*{Optimization Performance}

The optimization ran for 41 driver iterations, involving 41 model evaluations and 11 derivative evaluations. The total wall clock runtime was notably short, at 2 seconds and 757.5 milliseconds. The rapid execution time, coupled with the 'FAIL' exit status, suggests that the optimizer encountered an issue relatively quickly, possibly due to an initial infeasible state or numerical problems arising early in the process. Since the optimizer exited with a `FAIL` status, it did not converge to a solution that satisfied all constraints and minimized the objective. While SLSQP is generally a robust gradient-based algorithm, the issue here appears to stem from the fundamental problem formulation, likely an infeasible design space or highly constrained variables, rather than a deficiency in the algorithm itself.

\section*{Recommendations and Next Steps}

The optimization results are not reasonable due to the unmet primary constraint ($C_L = 2.0$) and the excessively high resulting drag coefficient. The optimizer's 'FAIL' status necessitates immediate investigation and a fundamental re-evaluation of the problem setup.

\subsection*{1. Investigate 'FAIL' Status}

The most critical initial step is to understand the precise reasons for the optimizer's failure. This could be due to an inherently infeasible design space, numerical instabilities within the solver, or overly aggressive constraints. A detailed review of the OpenMDAO log files for specific error messages is essential.

\subsection*{2. Review Geometric Constraints (Aspect Ratio)}

The problem statement fixed the wing area ($S = 100 \, \text{m}^2$) and span ($b = 10 \, \text{m}$). This combination yields an aspect ratio (AR) of $b^2/S = 10^2/100 = 1$. An aspect ratio of 1 is extremely low and is the dominant factor contributing to the poor aerodynamic performance and the optimizer's inability to meet the $C_L$ target efficiently. Achieving $C_L = 2.0$ with such a low aspect ratio wing is physically extremely challenging and will inherently lead to very high induced drag. This fixed, very low aspect ratio, while not an optimization constraint itself, is driving the suboptimal outcome. Future work should consider either relaxing these fixed geometric parameters or explicitly acknowledging that an AR=1 wing is generally unsuitable for efficient flight at $C_L = 2.0$.

\subsection*{3. Adjust Design Variable Bounds}

Many design variables, particularly `alpha`, `wing.taper`, and `wing.twist_cp`, hit their limits, indicating the optimizer was severely constrained. This suggests the available design space might be too restrictive for finding an efficient solution.
\begin{itemize}
    \item \textbf{Angle of Attack (alpha):} `alpha` hitting its upper limit of 15 degrees while failing to meet $C_L = 2.0$ implies the wing geometry cannot generate sufficient lift within the allowed AoA range. If the AR=1 constraint must be kept, relaxing `alpha`'s upper bound might give the optimizer more freedom, though this will likely exacerbate drag and flow separation issues for a Vortex Lattice Method (VLM) analysis.
    \item \textbf{Taper Ratio (`wing.taper`):} A taper ratio of 0.2 is very extreme. The physical and structural realism of such a design should be reviewed.
    \item \textbf{Twist Distribution (`wing.twist_cp`):} A constant 10-degree twist across the span is also an extreme outcome. Allowing for more nuanced twist distributions (e.g., using more control points or different twist functions) might provide the optimizer with better means to distribute lift more efficiently.
\end{itemize}

\subsection*{4. Re-evaluate Target CL}

Given the severely limited design space imposed by an AR=1 wing, a target $C_L$ of 2.0 might be physically infeasible for efficient flight. Consideration should be given to whether a lower target $C_L$ would be more appropriate for such a low aspect ratio wing, or if the problem definition itself needs to be revised to allow aspect ratio to be a design variable.

\subsection*{5. Check OpenAeroStruct Setup}

It is important to ensure that the mesh density, aerodynamic settings, and reference values within the OpenAeroStruct setup are appropriate for the intended analysis, especially considering such an extreme geometry and high angle of attack. Improper setup can lead to numerical inaccuracies.

\section*{Unrelated Observations (Highly Relevant to Problem Context)}

\subsection*{1. Fixed Geometric Parameters (AR=1)}
As highlighted in the recommendations, the fixed wing area ($S=100 \, \text{m}^2$) and span ($b=10 \, \text{m}$) inherently define an aspect ratio (AR) of 1. This is an extremely low aspect ratio. This fundamental geometric constraint, which was outside the optimizer's control, is the dominant factor leading to the observed poor aerodynamic performance and the optimizer's inability to meet the target $C_L$ efficiently. Designing an efficient wing with an AR of 1 for $C_L=2.0$ is nearly impossible with conventional wing theory and results in extremely high induced drag, as demonstrated by these results.

\subsection*{2. Manufacturability and Structural Concerns}
A taper ratio of 0.2 leads to a very narrow, pointed wingtip. Coupled with a constant 10-degree twist, such a wing might present significant manufacturing challenges and raises serious structural integrity concerns, particularly under the high aerodynamic loads implied by a 15-degree angle of attack and the high target $C_L$.

\subsection*{3. Vortex Lattice Method Limitations}
For an angle of attack of 15 degrees and a desired $C_L$ of 2.0 (even with an achieved $C_L$ of 0.993), the flow conditions are likely highly non-linear, with a significant potential for flow separation. The Vortex Lattice Method (VLM), being a linear aerodynamic method, may not accurately predict lift and drag under such extreme conditions. VLM generally tends to overestimate lift and underestimate drag in regions of significant flow separation. The high $C_D$ and the highly non-elliptical lift distribution already hint at these issues, suggesting that further analysis with a higher-fidelity method (e.g., RANS CFD) would be necessary to validate these results once a more reasonable preliminary design is obtained.

\end{document}